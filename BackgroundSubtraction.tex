\section{Background Subtraction}
\label{sec:BackgroundSubtraction}
Background subtraction

\subsection{$jets \rightarrow \gamma$ background estimation and subtraction}
The main background to W$\gamma$  W+jets which is ~XX\% of selected events and we don't see any way to significantly reduce it without reducing W$\gamma$ as well. Photon ID helps to reduce W+jets somewhat but there is still significant number of jets which are reconstructed as photons and pass all the photon ID criteria. DY+jets is another source of $jets \rightarrow \gamma$ background but this one is significantly supressed with W transverse mass cut [REFERENCE to subsection].

\subsubsection{Template Method Description}
The template method is used to estimate $jets \rightarrow \gamma$ background. The idea of the template method is the following. Suppose we have two varibles which are independent (at least for background events) $V_{fit}$ and $V_{sideband}$. We consider $V_{fit}$=$\sigma_{i \eta i \eta}^{photon}$ and $V_{sideband}$=$I_{ch-SCR}^{photon}$ and vice versa. 
We apply all the selection criteria on data except $\sigma_{i \eta i \eta}^{photon}$ and $I_{ch}^{photon}$. $\sigma_{i \eta i \eta}^{photon}$ is part of photon ID. $I_{ch-SCR}^{photon}$ is not part of photon ID but it is strongly correlated with $I_{ch}^{photon}$ which is part of photon ID. Then one needs to prepare the templates which would describe the distribution of true photons $T_{true}$ and of fake photons (e.g. jets reconstructed as photons) $T_{fake}$. In case of $V_{fit}$=$\sigma_{i \eta i \eta}^{photon}$ the true template is taken from FSR events of Z$\gamma$-selected dataset, in case of $V_{fit}$=$I_{ch-SCR}^{photon}$, random cone isolation method was used. Nominal cut of $V_{sideband}$ applied to prepare true-$\gamma$ templates. To prepare the fake template, distribution of $V_{fit}$ in sideband range of  $V_{sideband}$ is taken. Given that  $V_{fit}$ and $V_{sideband}$ are assumed to be independent, the $V_{fit}$ distribution of fake photons in nominal and sideband ranges of $V_{sideband}$ is the same. The leakage of true photons to sideband range of $V_{sideband}$ is subtracted based on signal-MC sample normalized to data luminocity.\\*
Then the nominal cut of  $V_{sideband}$ is applied on data and the distribution of $V_{fit}$ is fitted with $F(V_{fit})=N_{true} \cdot T_{true}(V_{fit}) + N_{fake} \cdot T_{fake}(V_{fit})$ where $N_{true}$ and $N_{fake}$ are fit parameters corresponding to fractions of true and fake photons in fitted data sample. To extract yield from the $N_{true}$, efficiency of the $V_{fit}$ is applied on the extracted yield. The efficiency is estimated from signal MC. [TODO: estimate it from Zg FSR events]. In case if $V_{fit}$ is variable used in photon ID and there is a bin boundary exactly at cut value, the yield can be extracted by summing over the true $\gamma$ template. It is possible if $V_{fit}$=$\sigma_{i \eta i \eta}^{photon}$ for most bins. However, the binning of template may be changed in case if there are not enough statistics in true or fake template and then it's possible that cut value of $V_{fit}$ will not coincide with any template bin boundary and thus the method of extracting yields by summing over histogram will become impossible and one will have to apply the efficiency of $V_{fit}$. In case of $V_{fit}$=$I_{ch-SCR}^{photon}$, the variable for fit is charged isolation computed with particle flow algorithm with super cluster removal [REFERENCE] but the variable used in the photon ID is computed without super cluster removal. These $I_{ch-SRR}^{photon}$ and $I_{ch}^{photon}$ are strongly correlated but they are not the same and thus to extract yield from the fit output, one has to apply efficiency of  $I_{ch}^{photon}$  used for photon ID.\\*
There were several studies performed to check whether our $\sigma_{i \eta i \eta}^{photon}$ and $I_{ch-SCR}^{photon}$ are independent and the studies showed that there is dependence between these two variables and the result vary significantly on sideband definition. \\*
TODO: include all our chiso vs sihih, chiso vs phopt and sihih vs phopt plots.\\*
We studied carefully whether this problem had place in approved CMS Z$\gamma$ analysis [REFERENCE] and found out that the problem was there too and we decided to follow their method how to deal with that. The idea of the approach is not to assume that the  $V_{fit}$ and $V_{sideband}$ are independent but to find the range of $V_{sideband}$ which would give the correct distribution for $V_{fit}$. We first do fits on the MC-mixture which mimics data and select the sideband range which would give the true and fake $\gamma$ yields the same as predicted by MC-truth information. Then we apply the same sideband range to data and perform fits on data. The fits are performed separately for barrel and endcap photons and separately for each pt bins. The systematic error according to this method [WILL BE] computed the same as it was done in Z$\gamma$ analysis mentioned above and WILL BE documanted in section [REFERENCE (systematics)].\\*

\subsubsection{Photon $I_{ch-SCR}$ fits}
Template method with $I_{ch-SCR}$ fits is our primary method for $jets \rightarrow \gamma$ background estimation. The signal template was obtained with Random Cone Isolation method [REFERENCE] and the background template was obtained from sideband of $\sigma_{i \eta i \eta}^{photon}$.
TODO: add here table for each with fit ranges and binning, sideband ranges, percent of signal to sideband leakage and whether the templates for different pt bins were combined, separately for tru and fake templates. The table must be produced automatically by script in latex format. Add 2D colorful plots for sideband variation for MC-mixture and data fits (probably, just one pt bin here and all bins to appendix) and plot the black dot on MC-mixture plot where the output coincide to MC-truth 

\subsubsection{Photon $\sigma_{i \eta i \eta}$ fits}
$\sigma_{i \eta i \eta}$ fits show bad performance probably due to especially high impact of the first bin in $I_{ch-SCR}$ distribution which is always thrown from the sideband. TODO: but we can document here all our studies, which are pretty much the same as for $I_{ch-SCR}$ fits. Also we can add Z$\gamma$ FSR selection here.

\subsubsection{Transfer function and fake distributions}
[Would be good if Yutaro describes here the method where he generates all the distributions with transfer function and if we had all the plots generated with this method instead of or in addition to data vs MC plots. Given that sihih fits show problems for all three of us, he may want to construct this transfer function and construct samples with true and fake rates which Katya and Lovedeep obtain with Random Cone Isolation method]

\subsection{$e \rightarrow \gamma$ background estimation and subtraction}
MC-based, not separated from true $\gamma$ background at the moment. The template method describe above doesn't distinguish between true $\gamma$ and $e \rightarrow \gamma$ contributions because electrons and photons reconstructed in ECAL would have the same shower shape and isolation distributions and thus $e \rightarrow \gamma$ proceeds as part of true $\gamma$ component given by template method.
For the electron channel, Z$\gamma$ is the main source of $e \rightarrow \gamma$ background. We managed to significantly supress the background by introducing Photon Pixel Veto, Photon conversion veto and Matching-GSF electron veto cuts. TODO: add quantitative effects of each cut to the table and dut to MC-based estimation this background doesn't exceed [QUESTION] X\% of signal. 
QUESTION: Does our overlap removal procedure takes care of removing this background from DYjets?
For the muon channel,  Z$\gamma$ is fully true $\gamma$ background and doesn't contribute to $e \rightarrow \gamma$. WW, WZ, ZZ, WW$\gamma$ etc. processes can have muon and electron misidentified as photon in final state and thus contribute to $e \rightarrow \gamma$ background in muon channel. Overall contrubution from these sources for muon channel doesn't exceed X\% of signal.
TODO: add table with all MC-based contributions, try to split between true-gamma and e to gamma. 

\subsection{True $\gamma$ background estimation and subtraction}
MC-based. Main source of true $\gamma$ background are Z$\gamma$, W$\gamma \rightarrow \tau \nu \gamma$, WW$\gamma$

\subsection{$jets \rightarrow lepton$ and $jets \rightarrow lepton + jets \rightarrow \gamma$}
Negligible (show from gamma+jets, diphoton and QCD MC samples)

